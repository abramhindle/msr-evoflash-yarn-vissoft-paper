\documentclass[times, 10pt,twocolumn]{article} 
\usepackage{latex8}
\usepackage{times}
\usepackage{xspace}
\usepackage{graphicx}
\usepackage{float}
\newcommand{\igH}[1]{\includegraphics[height=.9\textheight]{#1}}
\newcommand{\igHF}[1]{\includegraphics[height=\textheight]{#1}}
\newcommand{\igHhalf}[1]{\includegraphics[height=.4\textheight]{#1}}
\newcommand{\igHs}[1]{\includegraphics[height=.3\textheight]{#1}}
\newcommand{\igHthird}[1]{\includegraphics[height=.32\textheight]{#1}}
\newcommand{\igHh}[1]{\includegraphics[height=.5\textheight]{#1}}
\newcommand{\igW}[1]{\includegraphics[width=.9\textwidth]{#1}}
\newcommand{\igWh}[1]{\includegraphics[width=.5\textwidth]{#1}}
\newcommand{\igWhh}[1]{\includegraphics[width=.7\textwidth]{#1}}
\newcommand{\igWhalf}[1]{\includegraphics[width=.5\textwidth]{#1}}
\newcommand{\igWs}[1]{\includegraphics[width=.3\textwidth]{#1}}
\newcommand{\igWF}[1]{\includegraphics[width=\textwidth]{#1}}

\newcommand{\histodiff}{\emph{HistODiff}\xspace}
\newcommand{\crex}{\emph{C-REX}\xspace}
\newcommand{\loggen}{\emph{LogGen}\xspace}

\newcommand{\yarn}{\emph{YARN\xspace}}
\newcommand{\evoflash}{\yarn}
\newcommand{\YARN}{\yarn}
\newcommand{\postgresql}{\emph{PostgreSQL}\xspace}
\newcommand{\cvsup}{\emph{CVSup}\xspace}
\newcommand{\Subsection}[1]{\subsection{#1}}
%\documentstyle[times,art10,twocolumn,latex8]{article}

%------------------------------------------------------------------------- 
% take the % away on next line to produce the final camera-ready version 
\pagestyle{empty}

%------------------------------------------------------------------------- 
\begin{document}
\newcommand{\names}{Abram Hindle, ZhenMing Jiang, Walid Koleilat,
  Michael W. Godfrey, Richard C. Holt}
\author{
\names \\
University of Waterloo\\
\{ahindle,zmjiang,wkoleila,migod,holt\}@cs.uwaterloo.ca
}
\title{
YARN: Animating Software Evolution
}

\maketitle
\thispagestyle{empty}

\begin{abstract}

A problem that faces the study of software evolution is how to explore the
aggregated and cumulative effects of finely grained changes that occur
within a software system over time.  In this paper we describe an approach
to modeling, extracting, and animating the architectural evolution of a
software system.  We have built a prototype tool called \YARN (Yet Another
Reverse-engineering Narrative) that implements our approach;  \YARN mines
the source code changes of the target system, and then generates \YARN
``balls'' (animations) that a viewer can unravel (watch).  The animation
employs a static layout of the modules connected by animated edges that
model the changing dependencies.  Furthermore, the edges can be weighted by
the number of dependencies or the importance of the chance.  We demonstrate
our approach using the open source database system \postgresql as the target
system.

%-- \YARN animates changes in the dependencies between modules in a software
%-- system.  These dependencies are automatically extracted from a software
%-- repository (CVS) and then used to generate an animation. 
%-- 
%-- The animation features a static
%-- layout out of modules connected by animated edges. Edges are weighted by
%-- the number of dependencies or importance of the changes.  Finally, we
%-- demonstrate our approach using the open source database system
%\postgresql
%-- as the example system.

\end{abstract}



%------------------------------------------------------------------------- 

%Check YARN and conclusions
%Sell the approach then show the tool. APPROACH, abstract, into, conclusions
%Conclusion - improve
% * why is this interesting?
% Grand Overview
%  * Where is the problem
%  * What is the problem
%    * Visualization
%    * Diff Between Version
%    * Large Amount of data
% MAYBE PRESENT POINT IN BULLET FORM

%XWHAT IS THE RESEARCH PROBLEM
%X * Visualize Animate data
%X * Diff Between Revisions
%X * Visualize Large Data
%X * See cumulative effects of change (CREX Issue)
%X Discuss Edges (alpha channel and direction)
%X Add Ric and Mike

%XChange Text on Diagrams (white border, light grey background)
%Xs/evoFlash/Yarn/
%X * Explain yarn balls
%X * Yet Another Reverse engineering Narration
%X   * YARN BALLS
%X   * Yarn produces YARN balls which you unravel



%XAdd Related work or change Background to Background and Related Work
%X* Becker - Program Animation
%X* Claudia Riva ICSM 99
%X* Shrimp (animations between views)
%X* Jonathan Maletric 
%XExplain flipbook

%Reference Figures, integrate postgresql with evoflash, suggest correlation
% * better correlate
%Explain Static Architecture
%Explain Color
%Describe comparison diagram better
% * Or Compare two diagrams
%Edit
%Annotate Screenshot diagrams (paws?)
%Change Figure Colors

% Read
% Embeddable
% DIAGRAM: Describe animation NEED THIS
% Diagram overlayed animation
% X Flipbook

%CODE
% * show only changed edges
% * Color Functions
% * Evaluate Evolution?
%Writing up
% * 
%MSR IS 4 or 8 pages?
%Focus on EvoFlash
%HistODiff
%Color Functions


%Focus will be on EvoFlash and how it animates evolution
%Use postgresql examples but it might not be necessary to a tell a story
%IWPC 10 pages

\Section{Introduction}

\newcommand{\abram}[1]{\emph{(***ABRAM***: #1)}}
\newcommand{\graphfigfile}[1]{presentation/graph#1}
\newcommand{\graphlabel}[1]{fig:graph#1}
\newcommand{\yarnshots}{6}
\newcommand{\graphfig}[1]{
\begin{figure}[!b]
  \centering
  \igWh{\graphfigfile{#1}}
  \caption{\postgresql \YARN Flipbook shot #1/\yarnshots}
  \label{\graphlabel{#1}}
\end{figure}
}



%* Visualize Change over time
%* Share animations
%* Find important changes
%* map Arch to evolution

Successful software systems evolve in many ways and for many reasons:  bugs
are found and fixed, new features and deployment environments are requested
by users, and systems are refactored by developers to improve the internal
design.  However, the visualization and comprehension of change over time
is a problem that still faces the study of software evolution.  

For a given system, a developer may have access to static ``snapshots'' of
its software architecture and the internal dependencies.  These snapshots may
be hand drawn by an expert or even automatically generated from the source
code.  However, in practice these approaches are not well suited to the
task of comprehending how the system has evolved: hand drawn snapshots are
usually not maintained as the system ages, and automated architecture
visualization tools tend to emphasize static views of the current system
version.  

Emphasizing the changes to a system's architecture requires refocusing the
supporting tools toward calculating architectural deltas and then
representing them effectively to the user.  At the extractor level, this
might include performing fact extraction incrementally; in our case, we
establish a baseline architecture for the system, and the examine the CVS
commits that contain the code for the changes.  After analyzing the results
and reconciling the changes against the baseline, the resulting
architectural model of the system and its changes can be presented to the
user; we have chosen to use an animated visualization to do this.

%-- Thus the finer the granularity of change (releases, versions, commits), the
%-- more data there is to be evaluated. 
%-- 
%-- There are important architectural changes that occur at the finely-grained
%-- level of CVS commits, thus we wanted to show for a large software system
%-- such as \postgresql what changes had occurred. 
%-- 
%-- CVS commits often consist a feature addition that can drastically alter the
%-- dependencies between modules.

%-- One way to model change over time visually is to use animation. 

Animation is an intuitive and natural way to show change over time
visually.  We can start by showing the state of the architectural
dependencies within the system at a chosen baseline version, and then allow
the user to view the resulting changes progressively as animations to the
architectural visual model.  Given an interval of time we can take
advantage of cumulative views and show the differences in the context of
the whole system rather than just in the context of the instance. This is
often a problem with existing extractors because they may provide only only
a snapshot of a change and are not necessarily easy to reconcile to the
state of the system.

Tools to aid in studying the evolution of software systems typically deal
with very large data sets; often the data sets are so large that it is
impossible or impractical to view all of the data at once. Animation
enables us to traverse this rich and large information space and interpret
the data in a visual way rather than a textual or statistical way.

Some tools that animate graphs do so by moving the nodes around; instead,
we avoid moving nodes around and focus more on the dependencies between the
nodes.  We feel that by keeping the nodes in place, we allow for easier
comparison and allow one to perceive change easier, although it does
restrict how we can animate information.  Animation exploits the
temporality of the data in the repository and better illustrates the
dynamic behavior of the evolving dependencies between modules.  We employ
two approaches to representing the dependencies: the cumulative addition of
dependency and the difference of edge weights between changes.

We have included a demonstration of  \YARN and its generated \YARN Balls
(animation) in a flipbook-like form (figures \ref{fig:graph1},
\ref{fig:graph2}, \ref{fig:graph3}, \ref{fig:graph4}, \ref{fig:graph5},
\ref{fig:graph6}); thus, the reader can manually animate the printed Yarn
Ball like a flipbook.

\Subsection{Problem}

\graphfig{1}

%* Map arch to evolution
%* Will this find us interesting changes
%* arch to evolution
%* Visualize 

Our goal is to animate architectural aspects of finely grained change in a
useful and informative way.  To do so, several questions must first be
considered:  How do we show the architectural evolution of a software
system?  What can we hope to gain from such a visualization? Given the
large amount of data available what is a compact way to explore and exploit
it?  How can we show the cumulative effects of change?  Can we show change
effectively without getting ``lost''?  Can we provide useful parameters to
let the user fine tune the results?  For example, we considered:
\begin{description}
\item[Color:] In section \ref{sec:coloring} we discuss the various ways we
    can use edge color to indicate and highlight various flavors of change.
\item[Edge Width:] Edge width can be used to indicate the size of changes;
    unfortunately, it is hard to represent negative values with edge width.
    Edge widths can also be used to indicate age of changes; for example,
    edges can shrink as they age.
\item[Edge Existence:] Edges can be shown cumulatively or not.  That is,
    we can show a single change itself or we can show the cumulative effect of
    the change. Non-cumulative edges are problematic for negative weights,
    however.
\end{description}
These parameters can also represent more complex combinations of attributes,
rather than just a single attribute.  We can encode a particular
emphasis such as highlighting large architectural changes, or a focus on a certain kind of
behavior.

\Section{Related Research}

%* review (less general)

%RIGI
%SHRIMP
%LSEDIT
%PBS
%SOFTWARE EVOLUTION

There is a considerable amount of related work to this problem; much of the
research comes from the work done in the Mining Software Repositories
\cite{church04}, Software Evolution and Software Maintenance
\cite{software-mining} research communities.

One of the earliest uses of program visualization and animation is the well
known film ``Sorting Out Sorting'' by Baecker et al.\ \cite{sortingout},
which animated how values can be sorted by various algorithms.  More
recently, Gall et al.\ used 3D graphics to compare releases of a project
side by side \cite{GallJR99}.  Marcus et al.\ have also used 3D
visualization and animation of source code \cite{maletic}; their use of
animation was for interactivity rather to represent time.

Our architectural views are similar to those of Rigi \cite{rigi} and Shrimp
\cite{shrimp}.  In particular, we note that Shrimp makes use of animation
in its visualization, although it is used to support iterative navigation
rather than for representing change over time.

% * Baecker Two systems which produce animated representations of the
% execution of computer program

% 1   R. M. Baecker. Sorting out sorting (video). In Siggraph Video
% Review 7, 1981.
% 2   Baecker, R. (1982). Sorting Out Sorting, 30 minute colour sound film, available to order on video from http://www.utoronto.ca/~ic/media/vidcol/science.html
%  \cite{animated} \cite{sortingout}

%Lehman's was influential in relation to these research communities.
%Much of our analysis is related to work by
%Lehman such as ``Programs, Life Cycles and Laws of Software
%Evolution'' \cite{lehmans}. 

Finally, we note that our architectural model of \postgresql was adapted
from that of Dong et al.\ \cite{dong}.


\Section{Tools}

We first use \cvsup to obtain a local copy of the \postgresql CVS
repository.  Then \emph{C-REX} is used to extract the architectural
information from the repository.  Once extraction is done, we run
\emph{HistODiff}, which makes use of \emph{C-REX}'s output  to compute the
number of dependencies between subsystems, output the dependency graph, and
highlight architecturally important change.  Finally, all the dependency
graphs are read in by \YARN which produces the \YARN Balls, or animations
that can be played back by the user.  Figure \ref{fig:approach} illustrates
the extraction flow that we have used in our analysis.


\Subsection{C-Rex}


%SHORTEN THIS

We use \emph{C-REX} \cite{crex} as our fact extractor, as it has been
designed specifically for conducting historical architectural analysis.
It has several advantages over most architectural fact extractors.
First, traditional snapshot fact extractors, such as \emph{LDX} \cite{ldx},
\emph{RIGI} \cite{rigi}, and \emph{CPPX} \cite{cppx}, are designed to
retrieve architectural information from only one version of a system.
\emph{C-REX} is an evolutionary source code extractor; it extracts
information from version control systems and recovers architectural
information over a period of time.  Second, source code might not compile
properly due to the use of programming language dialects,
syntax errors, etc.  In this situation, most parser-based extractors will
fail, but \emph{C-REX} avoids fully parsing the source code by making use
of the \emph{ctags} source code tokenizing tool \cite{ctags}.  This makes
\emph{C-REX} more robust than most extractors.  Finally, most of extractors
operate on the preprocessed code or the object code.  Because of
compilation flags, a parser-based extraction results may contain
information specific to only a particular configuration;  \emph{C-REX}
operates on the original source code, therefore it can extract more
information relevant to software evolution than parser-based extractors.

\emph{C-REX} analyzes the main branch of a system's source code repository.
\crex extracts all the changes from each revision and groups revisions into
transactions.  It outputs two types of information: a \emph{Global Symbol
Table} and a set of \emph{Transaction Changes}.

The \textbf{Global Symbol Table} maps all of the programming language
entities ever defined during the history of software development to the
file locations where these entities are defined.  An entity can be of any C
language types, such as a macro, variable, function, struct, enum, etc.

\graphfig{2}

\textbf{Transaction Changes} list the entity changes committed by the same
author, at approximately the same commit time, with the same log message.
It contains the author's name, a unique hash value to identify this
transaction, the commit time, and the log message as well as detailed
entity changes.  An entity change can be one of three types depending on
the scope of the entity: \emph{modified} if the entity exists in both the
previous and current system revision, \emph{added} if it exists only in the
current revision, or \emph{removed} if it exists only in the previous
revision.

Within each changed entity, \crex keeps tracks of changes in entity types,
dependencies, and comment changes.  If the entity is a function, \crex also
tracks changes in parameters and return types.



\Subsection{HistODiff}


HistODiff performs many tasks:  it associates symbols to files; it resolves
references between changing architectures; it performs ``lifting'' of
architectural information; it filters the observed changes using heuristics
to identify key changes; it produces graphs for viewing; and it creates
reports of changes that are deemed interesting.  It makes the assumption
that each file is associated with a single module within the module
hierarchy.

\textbf{Symbol Mapping:} \histodiff resolves the context of multiple
symbols to functions and macros. The symbols are supplied by \crex's
output. The symbol mappings are important because the changes provided
indicate the symbols the changes depend upon.

\textbf{Architectural Mapping:} \crex produces a list of transactions where
entities such as functions, macros, and variables are added, deleted, and
modified. This list of transactions is used to update the architectural
dependency graph with the change in dependencies.

\textbf{Lifting:} The change information is ``lifted'' to the architectural
level, where the top-level subsystems and their dependencies are modeled.
%-- 
%-- A less cluttered graph is produced, it is the high level module graph,
%-- where only user defined subsystems are shown.  Essentially this is an lift
%-- operation of all the dependencies up to the user defined module level.
% This allows the output of all the graphs overtime as well as the
% difference between the 2 graphs.
Two kinds of graphs can be produced: a dependency graph between subsystems
over time, and a difference graph that shows the dependency changes
between subsystems before and after a given transaction.  The graphs have
directed weighted edges that indicate the number of calls between modules.

\textbf{Filtering:} In large projects such as \postgresql there will be
tens of thousands of 
transactions, but not all are architecturally
significant.  Large change transactions are often noticeable because they
affect many files or have a large line count, but small changes of one or
two lines can also add drastically alter the architecture and change the
dependencies between modules.  These are important changes to make note of,
because they could indicate some kind of architectural violation or
important feature addition.

Our filtering heuristics highlight transactions that affect the number of
dependencies between the top level subsystems and %\abram{Should that last 'and' be an 'or' instead?} 
that satisfy one of the following criteria:
\begin{itemize}
\item The transaction adds a dependency between two subsystems that were 
    previously unconnected.
\item The transaction removes a dependency between two subsystems that
    causes them to be unconnected.
\item The transaction doubles the number of dependencies between two subsystems.
\item The transaction reduces the number of dependencies between two
    subsystems by half.
\end{itemize}



\Subsection{YARN}

\begin{figure*}[htb]
  \centering
    \igHh{evoflash}
\caption{Screen-shot of \YARN with \postgresql}
\label{fig:evoscreen}
\end{figure*}


\graphfig{3}



%Show color functions
%* No colors, just highlight important changes
%* Highlight latest changes

The goal of \yarn (Yet Another Reverse-engineering Narrative) is to provide
a narrative animation; that is, the story of the evolution of a software
project over time.  \yarn uses the animation parameters and
\emph{HistODiff} output to generate \yarn Balls (animations) can be
unraveled (watched) by the user to learn about the history of the system's
architecturally significant changes.

%-- We like to call this spinning a yarn.


\yarn uses \histodiff's graph output to create a graphical animation of the
architectural changes of a system.  
%-- modules represented as nodes and their edges as dependencies.  
%-- The animations are created in SWF (Macromedia Flash) format, and can be
%-- embedded into webpages and viewed by most modern browsers.  
The thickness of the edges suggests how many dependencies exist between two
modules, we use the function $log^2(weight(u,v))$ to determine the edge's
thickness based on its weight.  The nodes are statically laid out so they
don't change position over time.  This allows for some sense of coherency
between changes.

%-- \yarn animates the changes to the dependencies of a graph over time.
%-- The modules of the software are laid out statically while the edges are
%-- animated. 

Edges are directed; when displayed, the edge of lesser weight is shown
inside of the edge going in the reverse direction.  Edges are also rendered
transparently, thus intersections of edges are both visible and visually
resolvable.

Edges can be animated into two different ways:



\begin{description}
\item[Cumulative view:] Edges exist over the entire time that there is a
    dependency between two modules. This view emphasizes the current state
    of the system and what edges are have been changed. 
    % what does relative means
\item[Non-cumulative view:] Edges only exist per each change. This view
    emphasizes what the actual changes are by removing the extra
    information.  
\end{description}

We have used several coloring schemes. Each uses color in a different way
in the animation to emphasize certain aspect of the changes over time.
Three such schemes are:

\begin{description}

\label{sec:coloring}

%\abram{I don't understand what this is saying.  Can you reword it please?}
\item[Color Changes on Modification:] This coloring algorithm changes
  the color of modified edges; this (obviously) serves to highlight
  edges that change.  When dependencies between two modules change,
  the edge is highlighted with the current color.  The current color
  is a function of time, thus the colors of newly modified edges are
  different colors than previous edges.  This allows older unchanged
  edges to maintain an out-dated color while edges that are changing
  display more current colors.  Edges which change frequently are
  noticeable due to their constantly changing color.  Although this
  scheme illustrates that change is occurring, this coloring scheme
  can lead to some ambiguity: edges that get brighter often look like
  they are decaying rather than changing.

\graphfig{4}

\item[Highlight and Decay:] Each edge that is modified is highlighted when
    a change occurs. It is highlighted by changing the color to a bright
    bold color; then over the period of a few changes the color of that
    edge decays back down to a neutral color. This color function
    emphasizes recent changes. Possible disadvantages include the decaying
    colors could look like new changes, also selecting an appropriate decay
    time could be difficult depending on how busy the graph is.

\item[Highlight the Important changes:] This coloring algorithm is much
    like highlight and decay algorithm except only the important changes
    are highlighted instead of just new changes. This view emphasizes
    changes that have been tagged as important.

\end{description}

All of the these algorithms assume that the edges are growing and shrinking
in width and that only the edges are being animated.

Edge widths can be displayed in several different ways:


\begin{description}
\item[Cumulative Width:] This edge width function is a scaling function such as
    $log^2(weight(u,v))$ (where $u$ and $v$ are modules and
    $weight(u,v)$ is the number of dependencies from $u$ to $v$ at the
    current step). Cumulative Width shows how many dependencies currently exist
    between the two different modules.

%    \abram{Once again, I don't understand this}
\item[Decaying Edge Width:] The older an edge gets the more it decays and the
    more it shrinks in width.  Over time an edge decays (shrinks) until it
    reaches a minimum width. When a change occurs it modifies the width of
    the edge back to its width according to the scaling function. 
\item[Edge Width as Age:] Instead of the number of dependencies, edge width
    alone indicates when the last change to the dependencies between two
    modules.
\end{description}

The changes animated are transactions, and one frame represents one
transaction. For instance, \postgresql had over 10,800 frames of animation.

In the top corner, the current date of the transaction is shown.
Underneath is the order of the revision. At the bottom a time-line
shows relationally where the transaction is with respect to the other
revisions. The time-line is click-able to allow jumping throughout the
evolution of the project.

The animations are created are vector graphics animations in SWF (Macromedia Flash) format,
 they can be embedded into webpages and viewed by most modern
browsers.  This could be used in the such hypermedia software evolution
systems like the  Software Bookshelf \cite{pbs} or SoftChange
\cite{dmgseke2004}. 

Modules can be laid out manually or automatically. Automatic layout
algorithms currently include radial or matrix layout. The radial layout is
useful for systems like \postgresql where there are many dependencies.
%-- , also we like the similarity it has to a ball of yarn.

Figures 
\ref{fig:graph1}, 
\ref{fig:graph2}, 
\ref{fig:graph3}, 
\ref{fig:graph4}, 
\ref{fig:graph5}, 
and 
\ref{fig:graph6} depicts 6 frames from \yarn (cropped) over time.
Figure \ref{fig:evoscreen} depicts a screen-shot of \yarn in action.

%\begin{figure}[htbp]
%  \centering
%\begin{tabular}{lr}
%    \igWh{presentation/graph1} &   \igWh{presentation/graph2} \\
%    \igWh{presentation/graph3} &   \igWh{presentation/graph4} \\
%    \igWh{presentation/graph5} &   \igWh{presentation/graph7} \\
%\end{tabular}
%\caption{Examples of graphs produced by \evoflash with \postgresql}
%\label{fig:evoflash}
%\end{figure}
%
%\begin{figure}[htbp]
%  \centering
%    \igWh{evoflash}
%\caption{Screen-shot of \evoflash with \postgresql}
%\label{fig:evoscreen}
%\end{figure}
%




\Section{Case Study of \postgresql}

%Show it more through time
%* SEE REPORT
%* A Story with evoflash

\Subsection{Architecture}

\postgresql is a well known open source DBMS that is in wide use.
\postgresql has a well defined layered architecture (Fig.\
\ref{fig:architecture}).  The use of layering provides many advantages
including the separation of concerns and abstraction.

The three layers are:

\begin{description}
\item[Client Interface Layer:] This layer accepts inputs from the users
    through a variety of user interfaces.  It submits the queries to the
    Backend layer below and returns the answers.  
\item[Backend:] This layer parses the user's query and presents
    it to the optimizer which attempts to produce the most
    efficient execution plan for the evaluation.  In order to execute the
    plan tree, this layer uses the I/O functions in the
    Data Store Layer.  
\item[Data Store Layer:] This layer deals with managing space on disk,
    where the data is stored.  Upper layers require this layer to write or
    read pages.  
\end{description}

Important sub modules of the layers include:

\graphfig{5}

\begin{description}
\item[LibPQ:] enables the client or user to ask queries of the RDBMS.
\item[System Control Manager:] handles authentication and starts and stops
    \texttt{postgres} processes.
\item[Traffic Cop:] delegates query jobs to the various query sub modules.
\item[Parser:] This modules tokenizes and parses SQL queries.
\item[Rewriter:] is the primary module for query rewriting queries which
    are recursive or can be optimized.
\item[Optimizer:] attempts to choose an optimal query plan structure for a
    given query.
\item[Executor:] executes the query plan (execution tree).

\item[Command:] is a query which doesn't need the Optimizer.
\item[Catalog:] is where the RDBMS stores the meta data, e.g information
    about tables, columns, values etc.
\item[Nodes:] are responsible for storing the queries in a specified common
    data structure called Nodes Structure.
\item[Util:] consists of utility procedures and routines that different
    modules use to do their jobs, especially Backend.
\item[Storage Manager:] is found inside of the data layer managing files
    and pages of the database.
\end{description}

\Subsection{A Visual Walk Through of PostgreSQL}

%XXXX THIS SECTION IS HORRIBLE REWRITE

We evaluate the history of architectural changes of \postgresql by manually
inspecting a few ``architecturally significant'' transactions.  \histodiff
flagged out about 100 transactions that it considered to be architecturally
significant; that is, they have either added or removed dependencies which
makes two subsystems connected or disconnected, or have significantly
increased or decreased the degree of coupling between subsystems.  
%\abram{I have absolutely no idea what the rest of the paragraph is supposed
%to mean.}
We divided the architecturally significant transactions into 3
intervals of equal number of transactions:

%XXXXXXXX PLEASE CHECK THIS XXXXXXXXXXXX
\begin{itemize}
\item 1996 to 1998: \postgresql is released as open source software.
    Portability and reimplementation of features such as ODBC are included
    (see figures \ref{fig:graph1} and \ref{fig:graph2}).



\item 1998 to 2000: \postgresql is still in flux, ODBC updates occurred and
    \postgresql was extended by the PL/pgSQL language (see figures
    \ref{fig:graph3} and \ref{fig:graph4}).  \item 2000 to 2005:
    \postgresql was being maintained, less features are added, more
    auditing and bug fixing occurs (see figures \ref{fig:graph5} and
    \ref{fig:graph6}).
\end{itemize}

There were many significant changes from 1996 to 2005, and we have
highlighted a few and have provided screenshots of the revisions.  These
screenshots help highlight the cumulative progression of architectural
dependency over time as well they highlight the architecturally significant
changes that occurred over time.

The first notable change is from figure \ref{fig:graph1} which shows the
result of adding PL/TCL. PL/TCL is a extension to \postgresql which allows
the TCL programming language to be used in stored procedures.  The 2
dependencies that are were changed are highlighted, these were between the
System Controller and Util (where PL/TCL was stored).

\Subsection{\postgresql: 1996 to 1998}

%INTRODUCTION (DURATION RELEASES...)
The theme of \postgresql from 1996 to 1998 was the transition from a closed
project to an open source project used by many users on many different
platforms ranging from UNIX to Win32. It covered releases 1.06 to 6.3.

\graphfig{6}

\begin{description}
\item[Security:] Many changes were related to authentication. These
    included host and plain-text password based authentication, and a check
    if \postgresql was running as a root user.
\item[Portability:] There were two kinds of portability issues dealt with:
    wrapping systems calls and distributing the output of tools such as
    bison and flex.  System call wrapping included wrapping signal with
    portable signal call code, as well as providing portable file open and
    close calls (for SunOS).  Flex and Bison output was added for systems
    which lacked compatible versions of the utilities (FreeBSD and AT\&T
    UNIX).
\item[Extensibility:] The server programming interface was added, it allows
    \postgresql to be used from languages such as C and TCL. As well PL/TCL
    was added which allows stored procedures to use TCL. Figure
    \ref{fig:graph1} depicts this change.
%\item[Optimization:] 
\item[Updated Code:] Old transactions code \emph{Time Travel} was removed.
    The old ODBC driver was updated with a new ODBC driver.  Figure
    \ref{fig:graph2} depicts this change. 
    There were also issues regarding the precedence of attributes from
    certain tables not being handled appropriately.
    The heap tuples code was optimized by in-lining the
    code via a macro.

\end{description}



\Subsection{\postgresql: 1998 to 2000}

The next period is from 1998 to  2000, which approximately
corresponds to 11 releases from Releases 6.4 to Release 7.0.3.

\begin{description}
\item[ODBC:] they have updated ODBC driver to version 0.0244.
\item[Extension:] The developers added a second procedural language
    PL/pgSQL and then later moved from the contributions directory into the official distribution. 
This change is depicted in figure
    \ref{fig:graph3}
\item[Enhancements: ] Developers improved the client/server asynchronous
    communication; added support for outer joins, and read/write locks;
    rewrote the \emph{Memory Management Process(mmgr)} utilities (this is
    depicted in figure \ref{fig:graph4}) for better handling of out of
    memory error; changed the header comments of functions.
    
\end{description}

%   \item[Graphical Snapshot: ]
%   The two graphs are snapshot taken from \evoflash indicating
%   the number of dependencies between subsystems at the beginning and end of this period.
% \begin{figure}[htbp]
%   \centering
% \begin{tabular}{lr}
%     \igWh{presentation/graph2} & \igWh{presentation/graph3} \\
% \end{tabular}
% \caption{Two snapshots of \postgresql in 1998 and 2001 by \evoflash}
% \end{figure}
%
% % Include diagram
% Comparing the two diagrams,
% edges between PARSER and LIBPQ, REWRITER and UTIL are thickening;
% new dependencies has added between EXECUTOR and SYSTEMCONTROLMANAGER.



\Subsection{\postgresql: 2000 to 2005}



The period 2000 to 2005 covered \postgresql releases 7.0.3 to 7.4.8.  This
period contains one third of the revisions that were flagged important in
the repository.  Large changes to the source included new implementations
of SQL statements, improving triggers, JOINS and dropped columns. Most of
the changes were maintenance and improvement of properties such as
robustness, security and performance.  Security improvements were spurred
by problems found by ``white hat hackers'', who found flaws related to
interrupt handling and critical sections. The changes that occurred due to
these flaws are depicted in figure \ref{fig:graph5}.  Better support for
Win32 signal handlers was also added (see figure \ref{fig:graph6}).  Figure
\ref{fig:pgsql05} depicts the changes made to the system starting from 2001
to 2005, also the larger changes are highlighted.


%\begin{figure}[htbp]
%  \centering
%\begin{tabular}{lr}
%   \igWh{presentation/graph4} &   \igWh{presentation/graph7} \\
%\end{tabular}
%\caption{Two snapshots of \postgresql in 2001 and 2005 by \evoflash}
%\label{fig:pgsql00}
%\end{figure}

\begin{figure*}[t]
  \centering
\begin{tabular}{lr}
    \igWh{presentation/graph4} & \igWh{analysis2005}\\
\end{tabular}
\caption{Important architectural changes done during the last 5
years} \label{fig:pgsql05}
\end{figure*}


\Section{Future Work}


%* Layout
%* Interactivity
%* Hierarchy
%* Metrics

The main extension to \YARN will be to its \YARN Ball user interface.  More
layout algorithms will be added to YARN.  We plan to add more interactivity
to the animation. This would include dragging and dropping modules as well
as expanding sub modules.  This would allow for hierarchical navigation of
sub modules.  Different views of the architecture are planned such as
source control view which shows which modules are coupled together per
commit.

Other work includes evaluating the use of animation for maintenance work.
We have yet to answer the question if this animation is useful to
developers or just researchers.


\Section{Conclusions}

%NEEDS MORE WORK

In conclusion we proposed and implemented a system which can extract the
dependency information from a repository and aggregate it into an
animation. These animations, or \YARN Balls, then can be embedded into
web pages or shared in order to communicate change based dependency
information about software projects.  Many different kinds of animations
can be produced.

The contributions of this paper thus included: an approach for animated the
evolution of a project's dependencies in a coherent static manner; a system
to view changes and the cumulative effects of the changes; provide a finely
grained view of the evolution of the project.






\begin{figure*}[p]
  \centering
  \igH{YARNApproach}
  \caption{Flow Control of Extraction with our tools}
  \label{fig:approach}
\end{figure*}

\begin{figure*}[p]
  \centering
    \igW{architecture}
\caption{Top Level Architecture of \postgresql}
\label{fig:architecture}
\end{figure*}


\bibliographystyle{latex8}

\bibliography{msr-evoflash}

\end{document}
